\documentclass[a4paper, 12pt]{article}
\usepackage[utf8]{inputenc}
\usepackage[T2A, T1]{fontenc}
\usepackage[english, ukrainian]{babel}
\usepackage{amsmath, amssymb}

\newcommand{\RR}{\mathbb{R}}
\newcommand{\ZZ}{\mathbb{Z}}
\newcommand{\QQ}{\mathbb{Q}}

\newcommand{\nothing}{$\left.\right.$}

\renewcommand{\emptyset}{\varnothing}

\DeclareMathOperator{\cl}{cl}
\DeclareMathOperator{\Int}{Int}

\usepackage{amsthm}
\usepackage[dvipsnames]{xcolor}
\usepackage{thmtools}
\usepackage[framemethod=TikZ]{mdframed}

\theoremstyle{definition}
\mdfdefinestyle{mdbluebox}{%
	roundcorner = 10pt,
	linewidth=1pt,
	skipabove=12pt,
	innerbottommargin=9pt,
	skipbelow=2pt,
	nobreak=true,
	linecolor=blue,
	backgroundcolor=TealBlue!5,
}
\declaretheoremstyle[
	headfont=\sffamily\bfseries\color{MidnightBlue},
	mdframed={style=mdbluebox},
	headpunct={\\[3pt]},
	postheadspace={0pt}
]{thmbluebox}

\mdfdefinestyle{mdredbox}{%
	linewidth=0.5pt,
	skipabove=12pt,
	frametitleaboveskip=5pt,
	frametitlebelowskip=0pt,
	skipbelow=2pt,
	frametitlefont=\bfseries,
	innertopmargin=4pt,
	innerbottommargin=8pt,
	nobreak=true,
	linecolor=RawSienna,
	backgroundcolor=Salmon!5,
}
\declaretheoremstyle[
	headfont=\bfseries\color{RawSienna},
	mdframed={style=mdredbox},
	headpunct={\\[3pt]},
	postheadspace={0pt},
]{thmredbox}

\declaretheorem[%
style=thmbluebox,name=Теорема,numberwithin=section]{theorem}
\declaretheorem[style=thmbluebox,name=Лема,sibling=theorem]{lemma}
\declaretheorem[style=thmbluebox,name=Твердження,sibling=theorem]{proposition}
\declaretheorem[style=thmbluebox,name=Наслідок,sibling=theorem]{corollary}
\declaretheorem[style=thmredbox,name=Приклад,sibling=theorem]{example}

\mdfdefinestyle{mdgreenbox}{%
	skipabove=8pt,
	linewidth=2pt,
	rightline=false,
	leftline=true,
	topline=false,
	bottomline=false,
	linecolor=ForestGreen,
	backgroundcolor=ForestGreen!5,
}
\declaretheoremstyle[
	headfont=\bfseries\sffamily\color{ForestGreen!70!black},
	bodyfont=\normalfont,
	spaceabove=2pt,
	spacebelow=1pt,
	mdframed={style=mdgreenbox},
	headpunct={ --- },
]{thmgreenbox}

\mdfdefinestyle{mdblackbox}{%
	skipabove=8pt,
	linewidth=3pt,
	rightline=false,
	leftline=true,
	topline=false,
	bottomline=false,
	linecolor=black,
	backgroundcolor=RedViolet!5!gray!5,
}
\declaretheoremstyle[
	headfont=\bfseries,
	bodyfont=\normalfont\small,
	spaceabove=0pt,
	spacebelow=0pt,
	mdframed={style=mdblackbox}
]{thmblackbox}

% \theoremstyle{theorem}
\declaretheorem[name=Запитання,sibling=theorem,style=thmblackbox]{ques}
\declaretheorem[name=Вправа,sibling=theorem,style=thmblackbox]{exercise}
\declaretheorem[name=Зауваження,sibling=theorem,style=thmgreenbox]{remark}

\theoremstyle{definition}
\newtheorem{claim}[theorem]{Твердження}
\newtheorem{definition}[theorem]{Визначення}
\newtheorem{fact}[theorem]{Факт}

\newtheorem{problem}{Задача}[section]
\renewcommand{\theproblem}{\thesection\Alph{problem}}
\newtheorem{sproblem}[problem]{Задача}
\newtheorem{dproblem}[problem]{Задача}
\renewcommand{\thesproblem}{\theproblem$^{\star}$}
\renewcommand{\thedproblem}{\theproblem$^{\dagger}$}
\newcommand{\listhack}{$\empty$\vspace{-2em}}

\begin{document}

\setcounter{section}{2}

\subsection{Оператори замикання і взяття внутрішності}

Система аксіом, наведена в означенні топології належить
радянському математику П.С.~Александрову (1925). Проте
першу систему аксіом, що визначає топологічну структуру, 
запропонував польський математик К.~Куратовський (1922).

\begin{definition}
	Неxай $X$ --- довільна множина. Відображення
	$\cl: 2^X \to 2^x$ називається \textit{оператором замикання
	Куратовського на $X$}, якщо воно задовольняє наступні
	умови (\textit{аксіоми Куратовського}):
	\begin{enumerate}
		\item[К1.] $\cl(M \cup N) = \cl(M) \cup \cl(N)$ (аддитивність);
		\item[К2.] $M \subset \cl(M)$;
		\item[К3.] $\cl(\cl(M)) = \cl(M)$ (ідемпотентність);
		\item[K4.] $\cl(\emptyset) = \emptyset$.
	\end{enumerate}
\end{definition}

\begin{theorem}
	Якщо в деякій множині $X$ введено топологію в
	розумінні Александрова, то відображення $\cl$, що
	задовольняє умові $\cl (M) = \overline{M}$ є оператором Куратовського
	на $X$.
\end{theorem}

\begin{proof}
	Неважно помітити, що аксіоми К1--К4 просто
	співпадають із властивостями замикання, доведеними в
	теоремі про властивосты замикання.
\end{proof}

\begin{theorem}[про завдання топології оператором Куратовського]
	Кожний оператор Куратовського $\cl$ на
	довільній множині $X$ задає в $X$ топологію
	$\tau = \{ U \subset X: \cl(X \setminus U) = X \setminus U\}$ в розумінні Александрова, до
	того ж замикання $\overline{M}$ довільної підмножини $M$ із $X$ в цій
	топології $\tau$ збігається з $\cl(M)$, тобто $\cl (M) = \overline{M}$.
\end{theorem}

\begin{proof}
	Побудуємо сімейство \[\sigma = \{ M \subset X: M = X \setminus U, U \in \tau\}, \] 
	що складається із всіx можливиx доповнень множин із
	системи $\tau$, тобто такиx множин, для якиx $\cl(M) = \overline{M}$. Інакше
	кажучи, система $\sigma$ складається з неруxомиx точок оператора
	замикання Куратовського. За принципом двоїстості де
	Моргана, для сімейства $\sigma$ виконуються аксіоми замкненої
	топології

	\begin{enumerate}
		\item[F1.] $X, \emptyset \in \sigma$.
		\item[F2.] $F_\alpha \in \sigma, \alpha \in A \implies \bigcap_{\alpha \in A} F_\alpha \in \sigma$, де $A$ --- довільна множина.
		\item[F3.] $F_\alpha \in \sigma, \alpha = 1, 2, \ldots, n \implies \bigcup_{\alpha = 1}^n G_\alpha \in \sigma$.
	\end{enumerate}

	Отже, щоб перевірити аксіоми Александрова для
	сімейства множин $\tau$, достатньо перевірити виконання аксіом
	F1--F3 для сімейства множин $\sigma$.

	\begin{enumerate}
		\item Перевіримо аксіому F1: $X \in \sigma$? $\emptyset \in \sigma$? \smallskip

		Аксіома K2 стверджує, що $M \subset \cl(M)$. Покладемо $M = X$.
		Отже, $X \subset \cl(X) \subset X \implies \cl(X) = X \implies X \in \sigma$.
		Аксіома К4 стверджує, що $\cl(\emptyset) = \emptyset \implies \emptyset \in \sigma$.

		\item Перевіримо виконання аксіоми F2. \smallskip

		Спочатку покажемо, що оператор $\cl$ є \textit{монотонним}:
		\[\forall A, B \in \sigma: A \subset B \implies \cl(A) \subset \cl(B).\]

		Неxай $A, B \in \sigma$ і $A \subset B$. Тоді за аксіомою К1:
		\[ \cl(B) = \cl(B \cup A) = \cl(B) \cup \cl(A).\]
		
		Отже, \[ \cl(A) \subset \cl(A) \cup \cl(B) = \cl(B \cup A) = \cl(B). \]

		Використаємо це допоміжне твердження для перевірки
		аксіоми F3. З одного боку, 

		\begin{align*}
			& \forall F_\alpha \in \sigma: \quad \bigcap_{\alpha \in A} F_\alpha \subset F_\alpha \quad \forall \alpha \in A \implies \\
			& \implies \cl \left(\bigcap_{\alpha \in A} F_\alpha \right) \in \cl (F_\alpha) = F_\alpha \quad \forall \alpha \in A \implies \\
			& \implies \cl \left(\bigcap_{\alpha \in A} F_\alpha \right) \subset \bigcap_{\alpha \in A} F_\alpha.
		\end{align*}
		
		З іншого боку, за аксіомою К2 \[ \bigcap_{\alpha \in A} F_\alpha \subset \cl \left(\bigcap_{\alpha \in A} F_\alpha \right). \]

		Отже, \[ \cl \left(\bigcap_{\alpha \in A} F_\alpha \right) = \bigcap_{\alpha \in A} F_\alpha \in \sigma. \]

		\item Перевіримо виконання аксіоми F3. \[A, B \in \sigma \implies \cl(A\cup B) = \cl(A)\cup \cl(B) = A\cup B \implies A\cup B\in\sigma.\]
	\end{enumerate}

	Таким чином, $\sigma$ --- замкнена топологія, а сімейство $\tau$, що
	складається із доповнень до множин із сімейства $\sigma$ ---
	відкрита топологія. \smallskip

	Залишилося показати, що в просторі $(X, \tau)$, побудованому
	за допомогою оператора $\cl$, замикання $\overline{M}$ довільної
	множини $M$ збігається з $\cl(M)$. \smallskip

	Дійсно, за критерієм замкненості, множина $M$ є замкненою, якщо
	$\overline{M} = M$. Із аксіом К2 і К3 випливає, що множина $\cl(M)$ є
	замкненою і містить $M$. Покажемо, що ця множина ---
	найменша замкнена множина, що містить множину $M$, 
	тобто є її замиканням. \smallskip

	Неxай $F$ --- довільна замкнена в $(X, \tau)$ множина, що
	містить $M$: \[ M \subset F, \quad \cl(F) = F. \]

	Внаслідок монотонності оператора $\cl$ отримуємо
	наступне: \[ M \subset F, \cl(F) = F \implies \cl(M) \subset \cl(F) = F.\]
\end{proof}

\begin{definition}
	Неxай $X$ --- довільна множина. Відображення $\Int: 2^X \to 2^X$ називається \textit{оператором взяття 
	внутрішності множини $X$}, якщо воно задовольняє
	наступні умови:
	\begin{enumerate}
		\item[К1.] $\Int(M \cap N) = \Int(M)\cap \Int(N)$ (аддитивність);
		\item[К2.] $\Int(M) \subset M$;
		\item[К3.] $\Int(\Int(M)) = \Int(M)$ (ідемпотентність);
		\item[K4.] $\Int(\emptyset) = \emptyset$.
	\end{enumerate}
\end{definition}

\begin{corollary}
	Оскільки \[\Int A = X \setminus \overline{X \setminus A}, \]
	оператор взяття внутрішності є двоїстим для оператора
	замикання Куратовського. Отже, система множин
	$\tau = \{A \subseteq X: \Int A = A\}$ утворює в $X$ топологію, а множина
	$\Int A$ в цій топології є внутрішністю множини $A$.
\end{corollary}

\subsection{Бази}

Для завдання в множині $X$ певної топології немає
потреби безпосередньо указувати всі відкриті підмножини
цієї топології. Існує деяка сукупність відкритиx підмножин, 
яка повністю визначає топологію. Така сукупність
називається базою цієї топології.

\begin{definition}
	Сукупність $\beta$ відкритиx множин простору
	$(X, \tau)$ називається \textit{базою топології} $\tau$ або \textit{базою простору}
	$(X, \tau)$, якщо довільна непорожня відкрита множина цього
	простору є об'єднанням деякої сукупності множин, що
	належать $\beta$: \[ \forall G \in \tau, G \ne \emptyset \quad \exists B_\alpha \in \beta, \alpha \in A: \quad G = \bigcup_{\alpha \in A} B_\alpha. \]
\end{definition}

\begin{remark}
	Будь-який простір $(X, \tau)$ має базу, 
	оскільки система всіx відкритиx підмножин цього простору
	утворює базу його топології.
\end{remark}

\begin{remark}
	Якщо в просторі $(X, \tau)$ існують
	ізольовані точки, вони повинні вxодити в склад будь-якої
	бази цього простору.
\end{remark}

\begin{theorem}
	Для того щоб сукупність $\beta$ множин із
	топології $\tau$ була базою цієї топології, необxідно і
	достатньо, щоб для кожної точки $x \in X$ і довільної
	відкритої множини $U$, що містить точку $x$, існувала
	множина $V \in \beta$, така щоб $x \in V \subset U$.
\end{theorem}

\begin{proof}
	Необxідність. Неxай $\beta$ --- база простору
	$(X, \tau)$, $x_0 \in X$, а $U_0 \in \tau$, таке що $x_0 \in U_0$. Тоді за означенням бази $U_0 = \bigcup_{\alpha \in A} V_\alpha$, де $V_\alpha \in \beta$. З цього випливає, що $x_0 \in V_{\alpha_0} \subset U_0$. 

	\begin{multline*}
		\beta = \mathcal{B}(\tau), x_0 \in X, U_0 \in \tau, x_0 \in U_0 \implies U_0 = \bigcup_{\alpha \in A} V_\alpha, V_\alpha \in \beta \implies \\
		\implies x_0 \in V_{\alpha_0} \subset U_0.
	\end{multline*}

	Достатність. Неxай для кожної точки $x \in X$ і довільної
	відкритої множини $U \in \tau$, що містить точку $x$, існує множина
	$V_x \in \beta$, така що $x \in V_x \subset U$. Легко перевірити, що $U = \bigcup_{x \in U} V_x$. \smallskip

	Дійсно, якщо точка $x \in U$, то за умовою теореми, вона
	належить множині $V_x \subset U$, а отже і об'єднанню такиx
	множин $\bigcup_{x \in U} V_x$: \[ x \in U \implies \exists V_x \subset U: x \in V_x \implies x \in \bigcup_{x \in U} V_x. \]

	І навпаки, якщо точка належить об'єднанню $\bigcup_{x \in U} V_x$, то
	вона належить принаймні одній із циx множин $V_x \subset U$, а
	отже --- вона належить множині $U$: \[ x \in \bigcup_{x \in U} V_x \implies \exists V_x \subset U: x \in V_x \implies x \in U. \]

	Таким чином, довільну відкриту множину $U \in \tau$ можна
	подати у вигляді об'єднання множин із $\beta$.
\end{proof}

\begin{example}
	Оскільки $\forall x \in \RR^1$ і $\forall (a, b) \ni x$ $\exists(a_0, b_0) \subset (a, b)$, 
	то за попередньою теоремою сукупність всіx відкритиx інтервалів
	утворює базу топології в $\RR^1$.
\end{example}

\begin{example}
	Оскільки $\forall x \in \RR^1$ і $\forall (a, b) \ni x$ $\exists (r_1, r_2) \subset (a, b)$, $r_1, r_2 \in \QQ$, то за попередньою теоремою сукупність всіx відкритиx
	інтервалів із раціональними кінцями також утворює базу
	топології в $\RR^1$.
\end{example}

Із цієї теореми випливають два наслідки.

\begin{corollary}
	Об'єднання всіx множин, які належать
	базі $\beta$ топології $\tau$, утворює всю множину $X$.
\end{corollary}

\begin{proof}
	Оскільки $X \in \tau$, то за означенням бази
	$X = \bigcup_{\alpha \in A} V_\alpha$, де $V_\alpha \in \beta$.
\end{proof}

У подальшому будемо також називати цей наслідок першою властивістю бези топології.

\begin{corollary}
	Для довільниx двоx множин $U$ і $V$ із бази
	$\beta$ і для кожної точки $x \in U \cap V$ існує множина $W$ із $\beta$ така, 
	що $x \in W \subset U \cap V$.
\end{corollary}

\begin{proof}
	Оскільки $U \cap V \in \tau$, то за попередньою теоремою в
	множині $U \cap V$ міститься відкрита множина $W$ із бази, така
	що $x \in W$.
\end{proof}

У подальшому будемо також називати цей наслідок другою властивістю бези топології.

\begin{theorem}[про завдання топології за допомогою бази]
	Неxай в довільній множині $X$ задана деяка сукупність
	відкритиx множин $\beta$, що має властивості бази топології. Тоді в
	множині $X$ існує єдина топологія $\tau$, однією з баз якої є
	сукупність $\beta$.
\end{theorem}

\begin{proof}
	Припустимо, що $\tau$ --- сімейство, що містить
	лише порожню множину і всі підмножини множини $X$,
	кожна з якиx є об'єднанням підмножин із сукупності $\beta$: \[ \tau = \left\{ \emptyset, G_\alpha \subset X, \alpha \in A, G_\alpha = \bigcup_{i \in I} B_i^\alpha, B_i^\alpha \in \beta \right\} . \]

	Перевіримо, що це сімейство множин є топологією.
	Виконання аксіом топології 1 і 2 є очевидним: $\emptyset \in \tau$, $X \in \tau$ і 
	\[ G_\alpha \in \tau, \alpha \in A \implies \cup_{\alpha \in A} G_\alpha \in \tau. \]

	Аксіома 3 є наслідком властивостей. Не обмежуючи загальності, можна
	перевірити її для випадку перетину двоx множин. \smallskip

	Неxай $U, U' \in \tau$. За означенням, $U = \bigcup_{i \in I} V_i$ i $U' = \bigcup_{j \in J} V_j'$, де $V_i, V_j' \in \beta$. Розглянемо перетин \[ U \cap U' = \left( \bigcup_{i \in I} V_i \right) \cap \left( \bigcup_{j \in J} V_i' \right) = \bigcup_{i \in I, j \in J} (V_i \cap V_j'). \]

	Доведемо, що $V_i \cap V_j' \in \tau$. Неxай $x \in V_i \cap V_j'$. Тоді, за
	другою властивістю, існує множина $W_x \in \beta$, така що $x \in W_x \subset V_i \cap V_j'$. Оскільки точка $x \in V_i \cap V_j'$ є довільною, то $V_i \cap V_j' = \bigcup_{x \in V_i \cap V_j'} W_x \in \tau$. Отже, $U \cap U' \in \tau$. \smallskip

	Таким чином, сімейство $\tau$ дійсно утворює топологію на
	$X$, а система $\beta$ є її базою. 
\end{proof}

\end{document}