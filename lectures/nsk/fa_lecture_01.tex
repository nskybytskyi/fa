\documentclass[a4paper, 12pt]{article}
\usepackage[utf8]{inputenc}
\usepackage[T2A,T1]{fontenc}
\usepackage[english, ukrainian]{babel}
\usepackage{amsmath, amssymb}

\newcommand{\RR}{\mathbb{R}}
\newcommand{\ZZ}{\mathbb{Z}}
\newcommand{\QQ}{\mathbb{Q}}

\newcommand{\nothing}{$\left.\right.$}

\renewcommand{\emptyset}{\varnothing}

\usepackage{amsthm}
\usepackage[dvipsnames]{xcolor}
\usepackage{thmtools}
\usepackage[framemethod=TikZ]{mdframed}

\theoremstyle{definition}
\mdfdefinestyle{mdbluebox}{%
	roundcorner = 10pt,
	linewidth=1pt,
	skipabove=12pt,
	innerbottommargin=9pt,
	skipbelow=2pt,
	nobreak=true,
	linecolor=blue,
	backgroundcolor=TealBlue!5,
}
\declaretheoremstyle[
	headfont=\sffamily\bfseries\color{MidnightBlue},
	mdframed={style=mdbluebox},
	headpunct={\\[3pt]},
	postheadspace={0pt}
]{thmbluebox}

\mdfdefinestyle{mdredbox}{%
	linewidth=0.5pt,
	skipabove=12pt,
	frametitleaboveskip=5pt,
	frametitlebelowskip=0pt,
	skipbelow=2pt,
	frametitlefont=\bfseries,
	innertopmargin=4pt,
	innerbottommargin=8pt,
	nobreak=true,
	linecolor=RawSienna,
	backgroundcolor=Salmon!5,
}
\declaretheoremstyle[
	headfont=\bfseries\color{RawSienna},
	mdframed={style=mdredbox},
	headpunct={\\[3pt]},
	postheadspace={0pt},
]{thmredbox}

\declaretheorem[%
style=thmbluebox,name=Теорема,numberwithin=section]{theorem}
\declaretheorem[style=thmbluebox,name=Лема,sibling=theorem]{lemma}
\declaretheorem[style=thmbluebox,name=Твердження,sibling=theorem]{proposition}
\declaretheorem[style=thmbluebox,name=Наслідок,sibling=theorem]{corollary}
\declaretheorem[style=thmredbox,name=Приклад,sibling=theorem]{example}

\mdfdefinestyle{mdgreenbox}{%
	skipabove=8pt,
	linewidth=2pt,
	rightline=false,
	leftline=true,
	topline=false,
	bottomline=false,
	linecolor=ForestGreen,
	backgroundcolor=ForestGreen!5,
}
\declaretheoremstyle[
	headfont=\bfseries\sffamily\color{ForestGreen!70!black},
	bodyfont=\normalfont,
	spaceabove=2pt,
	spacebelow=1pt,
	mdframed={style=mdgreenbox},
	headpunct={ --- },
]{thmgreenbox}

\mdfdefinestyle{mdblackbox}{%
	skipabove=8pt,
	linewidth=3pt,
	rightline=false,
	leftline=true,
	topline=false,
	bottomline=false,
	linecolor=black,
	backgroundcolor=RedViolet!5!gray!5,
}
\declaretheoremstyle[
	headfont=\bfseries,
	bodyfont=\normalfont\small,
	spaceabove=0pt,
	spacebelow=0pt,
	mdframed={style=mdblackbox}
]{thmblackbox}

% \theoremstyle{theorem}
\declaretheorem[name=Запитання,sibling=theorem,style=thmblackbox]{ques}
\declaretheorem[name=Вправа,sibling=theorem,style=thmblackbox]{exercise}
\declaretheorem[name=Зауваження,sibling=theorem,style=thmgreenbox]{remark}

\theoremstyle{definition}
\newtheorem{claim}[theorem]{Твердження}
\newtheorem{definition}[theorem]{Визначення}
\newtheorem{fact}[theorem]{Факт}

\newtheorem{problem}{Задача}[section]
\renewcommand{\theproblem}{\thesection\Alph{problem}}
\newtheorem{sproblem}[problem]{Задача}
\newtheorem{dproblem}[problem]{Задача}
\renewcommand{\thesproblem}{\theproblem$^{\star}$}
\renewcommand{\thedproblem}{\theproblem$^{\dagger}$}
\newcommand{\listhack}{$\empty$\vspace{-2em}}

\begin{document}

\section{Топологічні простори}

В курсі математичного аналізу уже розглядалися
поняття околу точки, відкритої і замкненої множин, точки
дотику, граничної точки, границі послідовності в просторі
$\RR$ тощо. Всі ці поняття визначалися за допомогою метрики
простору $\RR$ і відбивали певні властивості, притаманні
множинам, за допомогою яких ми могли описувати основну
концепцію цієї теорії --- близькість між точками. Адже саме
поняття близькості між точками (в розумінні малої відстані)
є базовим для таких головних понять математичного аналізу
як збіжність послідовностей і неперервність функцій. \medskip

Відносним недоліком цього підходу є очевидна
залежність від метрики, уведеної в просторі. Тому постало
питання, чи не можна побудувати більш абстрактну
конструкцію, за допомогою якої можна було б описати ідеї,
згадані вище. Серед дослідників цієї проблеми слід
відзначити французьких математиків M.~Фреше (1906),
M.~Рісса (1907--1908), німецького математика Ф.~Хаусдорфа
(1914), польського математика К.~Куратовського (1922) і
радянського математика П.~Александрова (1924). В
результаті досліджень цих та багатьох інших математиків
виникла нова математична дисципліна --- загальна
топологія, предметом якої є вивчення ідеї про неперервність
на максимально абстрактному рівні. \medskip

В цій та наступній лекціях ми введемо в розгляд ряд
важливих топологічних понять. Це дозволить нам вийти на
більш високий рівень абстракції і опанувати ідеї, що
пронизують майже всі розділи математики. Не буде
великим перебільшенням сказати, що в певному розумінні
топологія разом з алгеброю є скелетом сучасної
математики, а \textit{функціональний аналіз} --- це розділ
математики, головною задачею якого є дослідження
нескінченновимірних просторів та їх відображень.

\begin{definition}
	Нехай $X$ --- множина елементів, яку ми будемо
	називати носієм. \textit{Топологією} в $X$ називається довільна
	система $\tau$ його підмножин, яка задовольняє таким умовам
	(\textit{аксіомам Александрова}):
	\begin{enumerate}
		\item[A1.] $\emptyset, X \in \tau$;
		\item[A2.] $G_\alpha \in \tau$, $\alpha \in A$ $\implies$ $\bigcup_{\alpha \in A} G_\alpha \in \tau$, де $A$ --- довільна множина.
		\item[A3.] $G_\alpha \in \tau$, $\alpha = 1, 2, \ldots, n$ $\implies$ $\bigcap_{\alpha = 1}^n G_\alpha \in \tau$.
	\end{enumerate}
\end{definition}

Інакше кажучи, топологічною структурою називається
система множин, замкнена відносно довільного об'єднання
і скінченого перетину.

\begin{definition}
	Пара $T = (X, \tau)$ називається \textit{топологічним
	простором}.
\end{definition}

\begin{example}
	Нехай $X$ --- довільна множина, $\tau = 2^X$ ---
	множина всіх підмножин $X$. Пара $(X, \tau)$ називається
	простором з \textit{дискретною (максимальною) топологією}.
\end{example}

\begin{example}
	Нехай $X$ --- довільна множина, $\tau = \{\emptyset, X\}$.
	Пара $(X, \tau)$ називається простором з \textit{тривіальною
	(мінімальною, або антидискретною) топологією}.
\end{example}

Зрозуміло, що на одній і тій же множині $X$ можна ввести
різні топології, утворюючи різні топологічні простори.
Припустимо, що на носії $X$ введено дві топології --- $\tau_1$ і $\tau_2$.
Вони визначають два топологічні простори: $T_1 = (X, \tau_1)$ і
$T_2 = (X, \tau_2)$. \medskip

Говорять, що топологія $\tau_1$ є \textit{сильнішою}, або \textit{тонкішою},
ніж топологія $\tau_2$, якщо $\tau_2 \subset \tau_1$. Відповідно, топологія $\tau_2$ є
\textit{слабкішою}, або \textit{грубішою}, ніж топологія $\tau_1$. Легко бачити,
що найслабкішою є тривіальна топологія, а
найсильнішою --- дискретна.

\begin{remark}
	Mножина \textit{всіх} топологій не є цілком
	упорядкованою, тобто не всі топології можна порівнювати
	одну з одною. Наприклад, наступні топології (зв'язні
	двокрапки) порівнювати не можна: $X = \{a, b\}$,
	$\tau_1 = \{\emptyset, X, \{a\}\}$, $\tau_2 = \{\emptyset, X, \{b\}\}$.
\end{remark}

\begin{definition}
	Mножини, що належать топології $\tau$,
	називаються \textit{відкритими}. Mножини, які є доповненням до
	відкритих множин, називаються \textit{замкненими}.
\end{definition}

Наприклад, множина всіх цілих чисел $\ZZ$ замкнена в $\RR^1$.

\begin{remark}
	Топологія включає в себе всі відкриті
	множини. Водночас, треба зауважити, що поняття
	відкритих і замкнених множин не є взаємовиключними.
	Одна і та ж множина може бути одночасно і відкритою і
	замкненою (наприклад, $\emptyset$ або $X$), або ані відкритою, ані
	замкненою (множини раціональних та ірраціональних чисел
	в $\RR^1$). Отже, топологія може містити і замкнені множини,
	якщо вони одночасно є відкритими.
\end{remark}

Як бачимо, поняття відкритої множини в топологічному
просторі \textit{постулюється} --- для того щоб довести, що деяка
множина $M$ в топологічному просторі $T$ є відкритою, треба
довести, що вона належить його топології.

\begin{definition}
	Нехай $(X, \tau)$ --- топологічний простір, $M \subset X$.
	Топологія $(M, \tau_M)$, де $\tau_M = \{ U_M^{(\alpha)} = U_\alpha \cap M, U_\alpha \in \tau\}$,
	називається \textit{індукованою.} 
\end{definition}

\begin{definition}
	Топологічний простір $(X, \tau)$ називається
	\textit{зв'язним}, якщо лише множини $X$ і $\emptyset$ є замкненими й
	відкритими одночасно.
\end{definition}

\begin{definition}
	Mножина $M$ топологічного простору $(X, \tau)$
	називається \textit{зв'язною}, якщо топологічний простір $(M, \tau_M)$
	є зв'язним.
\end{definition}

\begin{example}
	Тривіальний (антидискретний) простір і
	зв'язна двокрапка є зв'яз\-ни\-ми просторами.
\end{example}

\begin{definition}
	Довільна відкрита множина $G \subset T$, що
	містить точку $x \in T$, називається її \textit{околом}.
\end{definition}

\begin{definition}
	Точка $x \in T$ називається \textit{точкою дотику}
	множини $M \subset T$, якщо кожний окіл $O(x)$ точки $x$ містить
	хоча б одну точку із $M$: \[\forall O(x) \in \tau: O(x) \cap M \ne \emptyset.\]
\end{definition}

\begin{definition}
	Точка $x \in T$ називається \textit{граничною точкою}
	множини $M \subset T$, якщо кожний окіл точки $x$ містить хоча
	б одну точку із $M$, що не збігається з $x$: \[\forall O(x) \in \tau: O(x) \cap (M \setminus \{x\}) \ne \emptyset.\]
\end{definition}

\begin{definition}
	Сукупність точок дотику множини $M \subset T$
	називається \textit{замиканням} множини $M$ і позначається $\overline{M}$.
\end{definition}

\begin{definition}
	Сукупність граничних точок множини
	$M \subset T$ називається \textit{похідною} множини $M$ і позначається
	$M'$.
\end{definition}

\begin{theorem}[про властивості замикання]
	Замикання
	задовольняє наступним умовам:
	\begin{enumerate}
		\item $M \subset \overline{M}$;
		\item $\overline{\overline{M}} = \overline{M}$ (ідемпотентність);
		\item $M \subset N \implies \overline{M} \subset \overline{N}$ (монотонність);
		\item $\overline{M \cup N} = \overline{M} \cup \overline{N}$ (адитивність);
		\item $\overline{\emptyset} = \emptyset$.
	\end{enumerate}
\end{theorem}

\begin{proof}
	\nothing
	\begin{enumerate}
		\item $M \subset \overline{M}$. \smallskip

		Нехай $x \in M$. Тоді $x$ --- точка дотику множини $M$. Отже,
		$x \in \overline{M}$.

		\item $\overline{\overline{M}} = \overline{M}$. \smallskip

		Внаслідок твердження 1) $\overline{M} \subset \overline{\overline{M}}$. Отже достатньо
		довести, що $\overline{\overline{M}} \subset \overline{M}$. Нехай $x_0 \in \overline{\overline{M}}$ і $U_0$ --- довільний окіл
		точки $x_0$. Оскільки $U_0 \cap \overline{M} \ne \emptyset$ (за означенням точки
		дотику), то існує точка $y_0 \in U_0 \cap \overline{M}$. Отже, множину $U_0$
		можна вважати околом точки $y_0$. Оскільки $y_0 \in \overline{M}$, то
		$U_0 \cap M \ne \emptyset$. Значить, точка $x_0$ є точкою дотику
		множини $M$, тобто $x_0 \in \overline{M}$.

		\item $M \subset N \implies \overline{M} \subset \overline{N}$. \smallskip

		Нехай $x_0 \in \overline{M}$ і $U_0$ --- довільний окіл точки $x_0$. Оскільки
		$U_0 \cap M \ne \emptyset$ (за означенням точки дотику) і $M \subset N$ (за
		умовою), то $U_0 \cap N \ne \emptyset$. Отже, $x_0$ --- точка дотику
		множини $N$, тобто $x_0 \in \overline{N}$. Таким чином, $\overline{M} \subset \overline{N}$.

		\item $\overline{M \cup N} = \overline{M} \cup \overline{N}$. \smallskip

		Із очевидних включень $M \subset M \cup N$ і $N \subset M \cup N$
		внаслідок монотонності операції замикання випливає, що
		$\overline{M} \subset \overline{M \cup N}$ і $\overline{N} \subset \overline{M \cup N}$. Отже, $\overline{M} \cup \overline{N} \subset \overline{M \cup N}$. З іншого
		боку, припустимо, що $x \notin \overline{M} \cup \overline{N}$, тоді $x \notin \overline{M}$ і $x \notin \overline{N}$. Отже,
		існує такий окіл точки $x$, у якому немає точок з множини
		$M \cup N$, тобто $x \notin \overline{M \cup N}$. Таким чином, за законом
		заперечення, $x \in \overline{M \cup N} \implies x \in \overline{M} \cup \overline{N}$, тобто
		$\overline{M \cup N} \subset \overline{M} \cup \overline{N}$.
		
		\item $\overline{\emptyset} = \emptyset$.
		Припустимо, що замикання порожньої множини не є
		порожньою множиною: $x \in \overline{\emptyset} \implies \forall O(x): O(x) \cap \emptyset \ne \emptyset$. Але
		$\forall N \subset X: N \cap \emptyset = \emptyset$. Отже, $\overline{\emptyset} = \emptyset$.
	\end{enumerate}
\end{proof}

\begin{theorem}[критерій замкненості]
	Mножина $M$
	топологічного простору $X$ є замкненою тоді і лише тоді,
	коли $\overline{M} = M$, тобто коли вона містить всі свої точки
	дотику.
\end{theorem}

\begin{proof}
	Необхідність. Припустимо, що $M$ --- замкнена
	множина, тобто $G = X \setminus M$ --- відкрита множина. Оскільки,
	$M \subset \overline{M}$, достатньо довести, що $\overline{M} \subset M$. Дійсно, оскільки $G$ --- відкрита множина, вона є околом кожної своєї точки.
	До того ж $G \cap M = \emptyset$. Звідси випливає, то жодна точка
	$x \in G$ не може бути точкою дотику для множини $M$, отже
	всі точки дотику належать множині $M$, тобто $\overline{M} \subset M$. \[ G = X \setminus M \in \tau \implies G \cap M = \emptyset \implies \overline{M} \subset M. \]

	Достатність. Припустимо, що $\overline{M} = M$. Доведемо, що
	$G = X \setminus M$ --- відкрита множина (звідси випливатиме
	замкненість множини $M$). Нехай $x_0 \in G$. З цього випливає,
	що $x_0 \notin M$, а значить $x_0 \notin \overline{M}$. Тоді за означенням точки
	дотику існує окіл $U_{x_0}$ такий, що $U_{x_0} \cap M = \emptyset$. Значить,
	$U_{x_0} \subset X \setminus M = G$, тобто $G = \bigcup_{x \in G} U_x \in \tau$.
\end{proof}

\begin{corollary}
	Замикання $M$ довільної множини $M$ із
	простору $X$ є замкненою множиною в $X$.
\end{corollary}

\begin{theorem}
	Замикання довільної множини $M$ простору
	$(X, \tau)$ збігається із перетином всіх замкнених множин, що
	містять множину $M$. \[ \forall M \in (X, \tau): \quad \overline{M} = \bigcap_\alpha F_\alpha, \quad F_\alpha = \overline{F_\alpha}, \quad M \subset F_\alpha. \]
\end{theorem}

\begin{proof}
	Нехай $M$ --- довільна множина із $(X, \tau)$ і
	$N = \bigcap_\alpha F_\alpha$, де $F_\alpha = \overline{F_\alpha}$, $M \subset F_\alpha$. \smallskip

	Покажемо включення $N \subseteq \overline{M}$: \[N = \bigcap_\alpha F_\alpha \implies (N \subseteq F_\alpha) \forall \alpha \implies (N \subseteq \overline{F_\alpha}) \forall \alpha.\]

	Оскільки $\{F_\alpha\}$ --- множина усіх замкнених множин, серед
	них є множина $\overline{M}$: $\exists \alpha_0: F_{\alpha_0} = \overline{M}$. Отже, \[ N \subseteq \overline{F_\alpha} \implies N \subseteq F_{\alpha_0} = \overline{M} \implies N \subseteq \overline{M}. \]

	Тепер покажемо включення $\overline{M} \subseteq N$. Розглянемо довільну замкнену множину $F$, що містить $M$: $\overline{F} = F$, $M \subset F$. Внаслідок монотонності замикання маємо: \[ \overline{F} = F \supset M \implies \overline{M} \subset \overline{F} = F \implies (\overline{M} \subset F_\alpha, F_\alpha = \overline{F_\alpha}) \forall \alpha \implies \overline{M} \subset N. \]

	Порівнюючи обидва включення, маємо \[ \overline{M} = \bigcap_\alpha F_\alpha. \]
\end{proof}

\begin{corollary}
	Замикання довільної множини $M$ простору
	$X$ є найменшою замкненою множиною, що містить
	множину $M$.
\end{corollary}

\begin{definition}
	Нехай $A$ і $B$ --- дві множини в топологічному
	просторі $T$. Mножина $A$ називається \textit{щільною} в $B$, якщо
	$\overline{A} \supset B$.
\end{definition}

\begin{remark}
	Mножина $A$ не обов'язково міститься в
	$B$: множина раціональних чисел є щільною в множині
	ірраціональних чисел і навпаки.
\end{remark}

\begin{definition}
	Якщо $\overline{A} = X$, множина $A$ називається \textit{скрізь
	щільною}. 
\end{definition}

\begin{definition}
	Mножина $A$ називається \textit{ніде не щільною},
	якщо вона не є щільною в жодній непорожній відкритій
	підмножині множини $X$.
\end{definition}

Mножина $A$ є щільною в кожній непорожній відкритій
множині, якщо $\forall U \in \tau, U \ne \emptyset: \overline{A} \supset U$, тобто кожна точка
множини $U$ є точкою дотику множини $A$. Отже,
$\forall x \in U: \forall O(x) \in \tau: O(x) \cap A \ne \emptyset$. Заперечення цього
твердження збігається з означенням ніде не щільної
множини. Формальний запис означення має такий вигляд.
\[ \exists U_0 \in \tau, U_0 \ne \emptyset: \overline{A} \not \supset U_0 \implies \exists x_0 \in U_0: \exists O(x_0) \in \tau: O(x) \cap A = \emptyset. \]

\begin{definition}
	Простір $T$, що містить скрізь щільну зліченну
	множину, називається \textit{сепарабельним}.
\end{definition}

\begin{example}
	В топології числової прямої множина всіх
	раціональних чисел $\QQ$ є щільною в множині всіх
	ірраціональних чисел $\RR \setminus \QQ$, і навпаки.
\end{example}

\begin{example}
	Найпростішими прикладами ніде не
	щільних множин є цілі числа просторі $\RR$ і пряма в просторі
	$\RR^2$.
\end{example}

\begin{example}
	Зліченна множина всіх раціональних чисел
	$\QQ$ є скрізь щільною у просторі $\RR$ , отже простір $\RR$ є
	сепарабельним.
\end{example}

З того, що $\overline{\QQ} = \RR$ і $\overline{\RR \setminus \QQ} =\RR$, зокрема, випливає, що $\QQ$ і
$\RR \setminus \QQ$ є ані відкритими, ані замкненими множинами.

\begin{example}
	Зліченна множина всіх поліномів з
	раціональними коефіцієнтами за теоремою Вєйєрштрасса є
	скрізь щільною в просторі неперервних функцій $C([a,b])$.
	Отже, простір $C([a,b])$ є сепарабельним. 
\end{example}

\end{document}